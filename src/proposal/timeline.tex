\section{Work Packages and Timeline}

\begin{table}[tb]
  \center
  \newcommand\cc{\blacksquare}
  \begin{tabular}{r @{\hspace{2em}}c c c c c c c c c c c c c c}
    \toprule
       & \multicolumn{13}{c}{Week} \\\cmidrule{2-14}
    WP &  2 &  3 &  4 &  5 &  6 &  7 &  8 &  9 & 10 & 11 & 12 & 13 & 14 \\
    \midrule
    1 & \cc & \cc & \cc & \cc                                                       \\
    2 &     &     &     & \cc & \cc & \cc & \cc                                     \\
    3 &     &     &     &     & \cc & \cc & \cc                                     \\
    4 &     &     &     &     &     &     &     & \cc & \cc & \cc & \cc & \cc & \cc \\
    5 &     &     &     &     &     &     &     &     &     &     &     & \cc & \cc \\
    6 &     &     &     &     &     &     &     &     &     &     & \cc & \cc & \cc \\
    \bottomrule
  \end{tabular}
  \caption{Timeline of Work Packages\label{tab:timeline}}
\end{table}

This project is divided into six distinct work packages. The first one is for gaining the necessary theoretical background for understanding the measurement-based computing model. This first phase involves literature search and reading on \emph{graph states}, which are essential to building a working knowledge of \emph{cluster states.} The second working package is about taking this theoretical framework and applying it to the model. With this phase, we plan on building a solid understanding of how measurement-based computation works, as well as some common applications of this computing schema. With the third package, we use the technical knowledge built by the previous two packages to write the introduction and method parts of the final report. The fourth working packages involves doing literature search seeking suitable algorithm candidates to re-formulate using a measurement-based model. After selecting the algorithm, we will re-formulate the problem as mentioned and conduct some resource requirement/performance analysis. The last two packages are for the preparation of our poster and the finalization of our final report.

The timelines and succinct descriptions of working packages are given in Tables \ref{tab:timeline} and \ref{tab:description} respectively.

\begin{table}[hbt]
  \begin{tabular}{r@{\hspace{2em}}p{0.8\linewidth}}
    \toprule
    WP & Description \\
    \midrule
    \renewcommand*{\arraystretch}{1.5}
    1 & Literature search and reading on theoretical background, including \emph{graph states} and \emph{cluster states.} \\
    2 & Thorough literature review on measurement-based quantum computing.\\
    3 & Writing of the introduction and methods parts of the written report.\\
    4 & Application of a suitable and well-known quantum algorithm to a measurement-based computing model.\\
    5 & Preparation of the poster.\\
    6 & Finalization of the written report.\\
    \bottomrule
  \end{tabular}
  \caption{Descriptions of Work Packages\label{tab:description}}
\end{table}

