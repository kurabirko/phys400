\section{Method}

\begin{wrapfigure}{r}{0.33\linewidth}
  \centering
  \tikz \graph { 1 -- { 2,3 } -- 4 };
  \caption{The diagram of graph \(G=\p{V, E}\) where \(G=\Set{1,2,3,4}\) and \(E=\Set{\p{1,2},\p{1,3}, \p{2,4}, \p{3,4}}\)\label{fig:generic_graph}}
\end{wrapfigure}

The main mathematical object used in this project is a graph state, so we provide a detailed definition. We start with the definition of a \emph{graph}. A graph \(G=\p{V, E}\) is the pair of a finite set \(V = \Set{1, 2, \dots, N}\) and a set \(E \subset V\times V\) \cite{clrs}. The set \(V\) is composed of elements that are called the \emph{vertices} of the graph. The elements of the set of paired vertices\(E\) are called the \emph{edges} of the graph. Graphs are usually represented visually using diagrams. One such example is given in Figure \ref{fig:generic_graph}. 

After establishing the definition of a graph, we define what a \emph{graph state} is. Let the unitary operator \(U_{ab}\) be a controlled \(\sigma_z\) operation applied between qubits \(a\) and \(b\).
\begin{equation}
  U_{ab} = 
  \begin{bmatrix}
    1 & 0 & 0 &  0 \\
    0 & 1 & 0 &  0 \\
    0 & 0 & 1 &  0 \\
    0 & 0 & 0 & -1 \\
  \end{bmatrix}
\end{equation}
The effect of \(U_{ab}\) applied on a qubit can be described as the following equation.
\begin{equation}
  U_{ab}\ket{+^{\spaceScript{a}}}\ket{+^{\spaceScript{b}}} = \frac{1}{\sqrt{2}}\p{\ket{0^{\spaceScript{a}}}\ket{+^{\spaceScript{b}}} + \ket{1^{\spaceScript{a}}}\ket{-^{\spaceScript{b}}}}
\end{equation}
So \(U_{ab}\) creates a maximally entangled Bell pair\cite{hein2006}. We use this constuct to create edges in a graph state. Let \(G=\p{G, E}\) be a graph. The graph state \(\ket{G}\) which corresponds to the graph \(G\) is the pure state with following wave function, where \(\ket+^V\) is the product state of \(\ket+\) for all vertices in the graph.
\begin{equation}
  \ket{G} = \prod_{\p{a, b}\in E} U_{ab} \: \ket{+}^V
\end{equation} 
So the preparation procedure for a graph state reads:
\begin{enumerate}
  \item Set the qubits at each vertex to \(\ket+\), the positive eigenstate of \(\sigma_z\).
  \item Apply \(U_{ab}\)\/ to each adjacent vertex in the graph.
\end{enumerate} 
An example graph state and its diagrammatic representation is given in Figure \ref{fig:graph_state}.

\begin{figure}[tb]
  \centering
  \begin{subfigure}[b]{0.4\linewidth}
    \begin{align*}
      \ket{G} 
        &= U_{23}U_{12}\ket{+}^{\otimes 3} \\
        & =\begin{aligned}[t]
          &\frac{1}{2} \ket{0^{\spaceScript{1}}} \ket{0^{\spaceScript{2}}} \ket{+^{\spaceScript{3}}} \\
          &+ \frac{1}{2} \ket{0^{\spaceScript{1}}} \ket{1^{\spaceScript{2}}} \ket{-^{\spaceScript{3}}} \\
          &+ \frac{1}{2} \ket{1^{\spaceScript{1}}} \ket{0^{\spaceScript{2}}} \ket{+^{\spaceScript{3}}}\\
          &- \frac{1}{2} \ket{1^{\spaceScript{1}}} \ket{1^{\spaceScript{2}}} \ket{-^{\spaceScript{3}}}
        \end{aligned}
    \end{align*}
    \caption{Wave Funciton Representation\label{fig:graph_state:wfn}}
  \end{subfigure}
  \hspace{5pt}
  \begin{subfigure}[b]{0.4\linewidth}
    \centering
    \begin{tikzpicture}[new set=import nodes]
  \begin{scope}[nodes={
    set=import nodes,
  }]
    \node (k1) at (0, 0) {$\ket{+^{\spaceScript{1}}}$};
    \node (k2) [right=1cm of k1] {$\ket{+^{\spaceScript{2}}}$};
    \node (k3) [below=1cm of k2] {$\ket{+^{\spaceScript{3}}}$};
  \end{scope}

  \graph [edges = {line width = 0.2mm}] {
    (import nodes);
    k1 -- ["$U_{12}$"]  k2 -- ["$U_{23}$"] k3;
  };
\end{tikzpicture}

    \caption{Diagrammatic Representation}
  \end{subfigure}
  \caption{Two different representations of the same graph state for the graph \(G=\p{\Set{1,2,3},\Set{(1,2),(2,3)}}\) \label{fig:graph_state}}
\end{figure}

The wave function representation of a graph state is not always the best for succinctly representing a graph. Figure \ref{fig:graph_state:wfn} shows this clearly. It also hides the entanglement patterns present in the graph state. To solve this issue, \emph{stabilizer} of this graph state is used to represent it textually. A stabilizer \(\symcal{S}\) is defined as a commutative subgroup of the Pauli group \(\symcal{P}^V\) that does not contain \(-\symbb{1}_V\) or \(\pm i\symbb{1}_V\). Any graph state \(\ket{G}\) for a given graph \(G = \p{G, E}\) satisfies the set of eigenvalue equations 
\begin{equation}
  K_a\ket{G} = +1 \ket{G}
\end{equation}
where
\begin{equation}
  K_a \coloneq  \sigma_x^a \prod_{b\in N_a} \sigma_z^b.
\end{equation} The symbol \(N_a\) denotes the set of adjacent vertices of the vertex \(a\), called its \emph{neighbourhood}. The commutative subgroup of \(\symcal{P}^V\) generated by the set of operators 
\begin{equation}
  \symcal{S} = \Set*{\;K_a \in \symcal{P}^V \given a \in V \; }
\end{equation}
is called the stabilizer of the graph state. Graph states are uniquely represented by their stabilizers. Note that graph state stabilizers expose the correlation patterns as they impose the following set of constraints to the measurement outcomes of their generators.
\begin{equation}
  \forall a\in V \; : \; m_x^a \prod_{b\in N_a} m_z^b = +1 
\end{equation} 

Now that we have provided a detailed account of what graph states are, we introduce a \emph{cluster state}. A cluster \(C\) is a \(d\)-dimensional array of qubits where each qubit is entangled with its neighbours. A simple graph can represent any cluster, so a cluster's quantum state \(\ket{C}\) is a graph state. This cluster construct will be used as a universal medium for quantum computation by applying measurements to its particles in a particular order and on a certain basis\cite{russendorf2001}.

This way of computation is called measurement-based quantum computing. This project will investigate how to construct algorithms using this formalism and the existing implementations of quantum algorithms for measurement-based quantum computers.