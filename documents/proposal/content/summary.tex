\section{Project Summary}

Quantum computing is a discipline that has sparked the interest of both physicist and computer scientists for its exciting performance implications. While the field has witnessed rapid progress in the development of new software, hardware technologies and efficient algorithms, the core principles at play have been quite stable. In almost every physical or abstract quantum computer, information encoding and processing occurs via unitary transformations while the result is written to a classical information storage via projective measurements. In this project we aim to investigate \emph{measurement-based quantum computing}, which is a different formulation for quantum computation that provides several advantages over quantum circuits. First, we will start with a review of \emph{graph states}, a general framework for studying the entanglement properties of qubits via graphs. Next, we are going to introduce the notion of a \emph{cluster state} which is a lattice like graph state that we will use to describe computation with. Cluster states are especially useful for this purpose since a careful arrangement of measurements being performed along a direction in the cluster can simulate the logic of any quantum program described by a quantum circuit. After describing the general computational framework, we are going to describe how common quantum computation constructs like Hadamard or phase operators can be realized by this schema. Finally, we are going to use these constructs to effectively describe a well-known quantum algorithm in terms of a measurement-based model. This description will allow us to compare different features of the gate-based model and measurement-based and gain further insight on the possible use cases.