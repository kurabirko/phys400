\section{Introduction}

Achieving an optimal runtime complexity is a fundamental challenge when it comes to designing scalable solutions for handling a large amount of data\cite{sipser13}. Recently, quantum computing has sparked the interest of both scientists and engineers' interest due to their asymptotical performance gain on some problems compared to classical computers. While the field has witnessed rapid progress in developing new software, hardware, and efficient algorithms, the core principles at play have been relatively stable. In almost every physical or abstract quantum computer, information encoding and processing occur via unitary transformations while the result is written onto classical storage via projective measurements\cite{Nielsen2009}. In this project, we aim to investigate \emph{measurement-based quantum computing}, a different formulation for quantum computation that provides several advantages over the circuit based model.

Measurement-based quantum computing was first proposed by Briegel and Russeldorf in 2000 \cite{Briegel_2001} as a general framework for universal computation by using the entanglement patterns of two-state particles. Their work was later extended to include a formulation based on stabilizers, whose use in quantum mechanics is mainly restricted to error correction \cite{Nielsen2009, quant-ph/9705052}. The structure of a measurement-based quantum computer was further generalized to encompass the use of graph states, which are used to represent qubits with Ising interaction patterns via simple graphs \cite{hein2006}. Recently, several algorithms formulated for measurement-based quantum computing have been suggested \cite{keith2014, debeaudrap2008theory, Fitzsimons2017}.

In this project, we start by reviewing the necessary mathematical background, where we introduce graphs and their relevant properties. We use these graph states as a general framework for studying the entanglement properties of special qubit systems, which can be represented by simple graphs. We use two different mathematical formulations for our exposition of graph states. One such way of formulating these objects is by using their interaction patterns. As graph states interact via specific Ising interactions, their states can be expressed by applying the well-known Ising interaction hamiltonians\cite{ichikawa2013}. The second formulation uses stabilizer formalism to achieve a more compact and transparent representation of the graph state and its entanglement patterns.

In the second phase of the project, we introduce the notion of a \emph{cluster state} which is a regular lattice-like graph state that we will use to describe computation. Following this introduction, we provide a mechanism to do universal computation. Lastly, we provide an example measurement-based algorithm, compare it to its circuit-based alternative formulation and discuss the possible benefits of using this formalism.