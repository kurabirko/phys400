\section{Results \& Discussion}

Our theoretical discussion provides an abstract machine to perform quantum computation using a measurement-based paradigm and implement circuit-based quantum algorithms using a measurement-based approach. We provide an implementation of Deutsch's algorithm below.

With a circuit based-formulation, Deutsch's algorithm takes a binary function \(f: \Set{0,1} \rightarrow \Set{0,1}\), and decides if it is constant or balanced. The oracle that implements the said function has the following form.
\begin{equation}
  U_f \, \ket{x}\ket{y} = \ket{x}\ket{y \oplus f(x)}
\end{equation} 
The oracle constructions that we have used are given in the Table \ref{tab:oracle}.

\begin{table}[htb]
  \center
  \begin{tabular}{llr}
    \toprule
    Type & Function & Oracle Unitary  \\
    \midrule
    Constant & \(f(x) = 0\) & \(\symbb{I}\)\\
    Constant & \(f(x) = 1\) & \(\sigma_x^1\) \\
    Balanced & \(f(x) = x\) & \(\cnot\) \\
    Balanced & \(f(x) = \neg x\) & \(\sigma_x^1\cnot\) \\
    \bottomrule
  \end{tabular}
  \caption{Deutsch's algorithm unitary oracle implementations}\label{tab:oracle}
\end{table}

The quantum circuit for the algorithm is given in Figure \ref{fig:deutsch_circuit}, based on \cite{Vallone2010}.

\begin{figure}[hb]
  \centering
  \begin{quantikz}
    \lstick{\(\ket+\)} & \qw & \qw & \gate[wires=2][2cm] {U_f} \gateinput{$x$} \gateoutput{$x$} & \gate{H} & \meter{} \\
    \lstick{\(\ket+\)} & \gate{R_z(\pi)} &\gate{H} & \gateinput{$y$} \gateoutput{$y \oplus f(x)$} & \qw & \qw 
  \end{quantikz}
  \caption{Quantum cicruit implementing Deutsch's algorithm}\label{fig:deutsch_circuit}
\end{figure}

We used Paddle Quantum\cite{Paddlequantum} library for simulating our MBQC circuits. The library includes an early version of an MBQC simulator which allows creating a graph state, measuring its vertex qubits and removing by-products. We used the procedure described in the previous section to convert the quantum circuit gate by gate to a series of measurement patterns and removed by-products on each step. We were able to distinguish between constant and balanced functions successfully.

