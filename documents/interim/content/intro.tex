\section{Introduction}

Achieving an optimal runtime complexity is a fundamental challenge when it comes to designing scalable solutions for handling a large amount of data\cite{sipser13}. Recently, quantum computing has sparked the interest of both scientists and engineers alike due to their asymptotical performance gain on some problems compared to the classical computers. While the field has witnessed rapid progress in developing new software, hardware and efficient algorithms, the core principles at play have been relatively stable. In almost every physical or abstract quantum computer, information encoding and processing occur via unitary transformations while the result is written onto classical storage via projective measurements\cite{Nielsen2009}. In this project, we aim to investigate \emph{measurement-based quantum computing}, a different formulation for quantum computation that provides several advantages over the circuit based model.

Measurement-based quantum computing was first proposed by Briegel and Russeldorf in 2000 \cite{Briegel_2001} as a general framework for universal computation  by using the entanglement patterns of two-state particles. Their work was extended to include a formulation based on stabilizers, whose use in quantum mechanics have mostly been restricted to error correction \cite{Nielsen2009, quant-ph/9705052}. The structure of a measurement-based quantum computer was later generalized into the notion of graph states, which are used to represent qubits with Ising interraction patterns via simple graphs \cite{hein2006}. Later, several algorithms formulated for measurement-based quantum computing have been suggested \cite{keith2014, debeaudrap2008theory}.

In this project, we started with a review of the mathematical background necessary. Graphs and their relevant proerties were introduced which were used to formulate the method of computation. Later, graph states were introduces as a general framework for studying the entanglement properties of systems of qubits which can be represented by simple graphs. The introduction of graph states involves two different formulations. One way to formulate these objects is by interaction patters. As graph states interract by specific Ising interraction patterns, their states can be found by applying the well known Ising interraction hamiltonians\cite{ichikawa2013}. The second formulation is using the stabilizer fromalism to achieve a more compact and transparent representation of the graph state and its entanglement patterns.

In the second phase of the project, we introduce the notion of a \emph{cluster state} which is a lattice-like graph state that we will use to describe computation with. Following this introduction, we are going to provide a mechanism to do universal computation. Lastly, we are going to provide a few examples for measurement-based algorithms, compare them to their circuit-based alternatives and discuss the possible benefits of using this formalism.